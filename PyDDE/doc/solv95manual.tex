\documentclass[10pt] {article}
\usepackage{epsf}
\usepackage{latexsym}
\setlength{\textheight}{22.5cm} 
\setlength{\textwidth}{16cm}
\setlength{\oddsidemargin}{-5mm} 
\setlength{\topmargin}{0.1cm}
\setlength{\evensidemargin}{-5mm}

%% Definitions
\newcommand {\hide}[1] {\typeout{ #1 }}
%Comment out to print all
%\newcommand {\hide}[1] {{\it #1 }}
%Comment out to hide some
\newcommand{\beq}{\begin{equation}}
\newcommand{\eeq}{\end{equation}}
\newcommand{\dif}[2]{\frac{{\rm d} #1}{{\rm d} #2}}
\newcommand{\ildif}[2]{{\rm d} #1/{{\rm d} #2 }}
\newcommand{\ilpdif}[2]{\partial #1/{\partial #2 }}
\newcommand{\pdif}[2]{\frac{\partial #1}{\partial #2}}
\newcommand{\pddif}[3]{\frac{\partial^2 #1}{\partial #2 \partial #3}}
\newcommand{\ilpddif}[3]{\partial^2 #1/{\partial #2 \partial #3}}
\newcommand{\bm}[1]{\mbox{\boldmath $#1$}}
\newcommand{\comb}[2]{\left (\begin{array}{c}{#1}\\{#2}\end{array}\right )}
\newcommand{\gfrac}[2]{\mbox{$ { \textstyle{ \frac{#1}{#2} }\displaystyle}$}}

% comment out next line unless double spacing needed
%\renewcommand{\baselinestretch}{2}


\begin {document}
\begin{center}
{\Large \bf Solv95: a numerical solver for systems of delay \\
differential equations with switches}\\
\vspace{0.3cm}
Simon N. Wood,\\ Mathematical Institute, North Haugh, St. Andrews, Fife KY16 9SS, UK.\\ {\tt %%@
snw@st-and.ac.uk}
\end{center}

\tableofcontents

\section{Introduction}

Solv95 is designed to allow numerical solution of systems of ordinary and delay differential %%@
equations, which may have state variable discontinuities (switches). To use it you need a C/C++ %%@
compiler/linker capable of producing 32 bit Windows executables with a graphical user interface. %%@
Suitable compiler/linkers are supplied commercially by Borland or Microsoft (these come with %%@
quite useful debugging facilities), or you can use GNU C/C++ for Windows, which is available %%@
without charge (but has slightly less user friendly debugging tools) . To solve a set of %%@
equations you must edit a template C file to specify the model. This is then compiled and linked %%@
to the numerical and graphical routines which will solve the model and display the results. %%@
solv95 was originally designed to solve population dynamic models.

Solv95's advantages are that the numerical analysis is particularly carefully done with respect %%@
to solution of delay differential equations, it is quite fast relative to other packages (using %%@
efficient adaptive timestepping), its graphics use only the Windows API (built in to Windows) so %%@
you don't need extra toolboxes or add-ins to use it and it's open source. Switches are treated in %%@
a less rigorous manner (but I don't know of a package that does any better). Solv95's %%@
disadvantages are that it doesn't do anything except solve your equations, it does little to %%@
check that the user's code is error free, and some of the Windows programming is inelegant (it %%@
was my first attempt at Windows programming and was then ported from Windows 3.1 to OS/2 to %%@
Win32).

This manual is not designed to be comprehensive, but is designed to explain how to code up a %%@
model and how to compile and link it once you have done so. Under the assumption that most users %%@
will start with one of the supplied example models, I will first describe compiling and linking. %%@
I will then explain how models are specified in detail 
and provide examples of how to do so. Finally I will briefly discuss the algorithms used.   

The package is provided on condition that the following be accepted: (i) The package comes with %%@
no warrenty and remains the property of the author. (ii) The package and compiled models may be %%@
re-distributed, provided that no charge is made for doing so, original authorship is fully %%@
acknowledged and these conditions are accepted by the recipient.   

\section{Compiling and Linking models}

The package has been successfully used with Borland C/C++, Microsoft Visual C/C++ and GNU gcc %%@
ported to Windows. I'll describe how to use \verb+gcc+ first.

\subsection{Using {\tt gcc}}

If you don't already have a port of \verb+gcc+ (the gnu compiler suite) for Windows then you need %%@
to get it. I used the very excellent \verb+mingw32+ (stands for Minimalist GNU Win32) downloaded %%@
from:\\
\verb+ftp://ftp.xraylith.wisc.edu/pub/khan/gnu-win32/mingw32/egcs-1.1.1/+ \\
But for more recent information checkout:\\
\verb+http://www.xraylith.wisc.edu/~khan/software/gnu-win32/egcs.html+\\
An excellent tutorial on \verb+gcc+ for Windows is to be found at:\\
\verb+http://www.geocities.com/Tokyo/Towers/6162/gcc.html+.\\
\verb+mingw32+ will give you a suite of command line tools, including a debugger \verb+dbx+ and %%@
the GNU compiler \verb+gcc+.

Suppose that your model definition file is called {\tt model.c}. To compile and link solv95:
\begin{enumerate}
\item Open an {\tt msdos} prompt, and change to the directory containing all the {\tt .c}, {\tt %%@
.h} and {\tt .rc} files.
\item Compile the source code for your model and the package with the line:\\
{\tt gcc -c solv95.c ddeq.c model.c} this will yield object files {\tt solv95.o, ddeg.o} and {\tt %%@
model.o}
\item Compile the windows resources that are needed:\\
{\tt windres -i solv95r.rc -o solv95r.o}.
\item Link the object files into an executable:\\
{\tt gcc -o solv95.exe ddeq.o solv95.o template.o solv95r.o -mwindows} 
\end{enumerate}
Now type {\tt solv95} at the command prompt to run the program. 

You may find it convenient to compile {\tt solv95.c}, {\tt ddeq.c} and {\tt solv95r.rc} just once %%@
and then combine them into a library file with the command:\\
{\tt ar -ru solv95.a ddeq.o solv95.o solv95r.o}\\
Once this has been done the line \verb+gcc -c model.c+ compiles the model, while:\\
\verb+ gcc -o solv95.exe model.o solv95.a -mwindows+ \\
will link it (note that the \verb+.a+ file must come after the \verb+.a+ files in the argument %%@
list).

\subsection{Borland or Microsoft}

I only have access to rather old versions of these products, so these instructions will be quite %%@
vague. Assume that your model definition file is called {\tt model.c}. Under both products you %%@
need to include {\tt solv95.c}, {\tt ddeq.c}, {\tt solv95r.rc} and {\tt model.c} in a `project'. %%@
You then `make' or `build' the project to compile and link and should get an executable which can %%@
be run.  

\section{Defining models}

To define a model you need to edit various functions in a standard template file. You may also %%@
need to use some standard functions. This section explains these. The following table summarises %%@
what you need to know about for each class of models that the package can deal with (SDDE stands %%@
for delay differential equations with switches):

\begin{center}

\begin{tabular}{|l|c|c|c|}
\hline
Functions to define & ODE & DDE & SDDE \\
\hline
{\tt initcons()} & $\bullet$&$\bullet$&$\bullet$ \\
{\tt switchfunctions()}& & &$\bullet$ \\
{\tt map()} & $$&$$&$\bullet$ \\
{\tt grad()} & $\bullet$&$\bullet$&$\bullet$ \\
{\tt storehistory()} & $$&$\bullet$&$\bullet$ \\
{\tt initstate()} & $\bullet$&$\bullet$&$\bullet$ \\
{\tt statescale()} & $\bullet$&$\bullet$&$\bullet$ \\
{\tt initout()} & $\bullet$&$\bullet$&$\bullet$ \\
\hline
Functions to use & && \\
\hline
{\tt pastvalue()} & &$\bullet$&$\bullet$ \\
{\tt pastgradient()} & &$\bullet$&$\bullet$ \\
\hline
\end{tabular}
\end{center}

Note that (i) functions that are not needed for a model class can be left empty, and (ii)
in everything that follows (without loss of generality) I will assume that integration is with %%@
respect to time.

\subsection{state variables and constants}

The program aims to solve problems of the general form:
\begin{eqnarray*} 
\dif{s_0}{t} & = & g_0 ({\bf s}(t),{\bf s}(t-\tau_0), {\bf s}(t-\tau_1), \ldots,t)\\
\dif{s_1}{t} & = & g_1 ({\bf s}(t),{\bf s}(t-\tau_0), {\bf s}(t-\tau_1), \ldots,t)\\
\dif{s_2}{t} & = & g_2 ({\bf s}(t),{\bf s}(t-\tau_0), {\bf s}(t-\tau_1), \ldots,t)\\
. & . & ~~~~~~~~~. \\
. & . & ~~~~~~~~~. 
\end{eqnarray*}
subject to initial conditions on the $ s_i$'s, and with the possible addition of some %%@
discontinuities in the $ s_i$'s. The $ s_i$'s are the {\it state variables } of the problem and $ %%@
{\bf s}(t)$ is used to mean the vector of all state variables at time $ t$, so $ {\bf s}(t-\tau)$ %%@
is the vector of state variables time $ \tau $ ago. To define a model the user must specificy the %%@
functions $ g_i$ which return the gradients of $ s_i$ w.r.t. time given the state variables at %%@
time $ t$ and at a series of lagged times. The user must also specify the starting values for all %%@
state variables.

Within the program the state variables are passed to and from functions as an array \verb+s[]+. %%@
The functions defining the gradients of the state variables may depend on various user defined %%@
constants (as may the initial values) - these constants are passed around in an array \verb+c[]+. %%@
All arrays start at element \verb+0+, not element \verb+1+. Elements of \verb+s[]+ must not be %%@
modified anywhere, except in \verb+map()+ and \verb+initstate()+. If you modify elements of %%@
\verb+c[]+ then the changes will be permanent until you next change them.

\subsection{\tt initcons(no\_vars,no\_cons)}

The arguments of this function are both pointers to integers.

The purpose of this function is to specify the number of state variables and number of constants %%@
that the problem has. For example if your problem has one state variable and 2 constants, then %%@
the function should read:
\begin{verbatim}
{ *no_cons=2;
  *no_vars=1;
}
\end{verbatim}

\subsection{\tt switchfunctions(sw,s,c,t)}

The arguments of this function are all pointers to double, except \verb+t+, the time, which is a %%@
double. \verb+sw+ is an array in which to return switchfunction values; \verb+s[]+ is the array %%@
of state variables; \verb+c[]+ is the array of constants. Elements of \verb+s[]+ must not be %%@
altered in this function.

Switches are discontinuous changes in state variables. Switchfunctions are functions that pass %%@
through zero, with a negative time derivative at the time that a switch is to occur. Switches %%@
have an index (starting at zero). Users define the switchfunctions in the %%@
\verb+switchfunctions()+ routine by setting the values of the switchfunctions in the array %%@
\verb+sw[]+. For example, if the 0th switch is to occur every 33.2 time units starting at time %%@
11.1 the following line could be used:\\
\verb+sw[0]= - sin(6.28318530718*(t-11.1)/33.2);+\\
As another example, if switch 1 is to occur just once at time 55.67 then the following line is %%@
appropriate:\\
\verb+sw[1]=55.67-t;+\\
Switchfunctions can depend on constants and state variables, although some care may be required %%@
to ensure a well defined model when switchfunctions are state variable dependent.

When the $i^{th}$ switchfunction passes through zero, from positive to negative, then integration %%@
is suspended (in an orderly fashion) and the routine \verb+map()+ is called with $ i$ passed as %%@
argument \verb+swno+. \verb+map()+ is where the user defines switches.

\subsection{\tt map(s,c,t,swno)}

The arguments of this function are (pointer to double) \verb+s[]+ the array of state variables; %%@
(pointer to double) \verb+c[]+  the array of constants; (double) \verb+t+ the time; (integer) %%@
\verb+swno+, the index of the switch whose switchfunction is passing through zero (with negative %%@
slope).

\verb+map()+ is where you must define the switches, i.e. the discontinuous changes in state %%@
variables \verb+s[]+ triggered by a switchfunction descending through zero (see {\tt %%@
swichfunctions()} above). For example, in the following code state variables 0 and 1 are doubled %%@
by switch 0 every time it is triggered, while state variable 0 is multiplied by \verb+c[0]/t+ by %%@
switch 1 whenever it is triggered:
\begin{verbatim}
if (swno==0) {s[0]*=2;s[1]*=2;} else
if (swno==1) {s[0]*=c[0]/t;}
\end{verbatim} 

\subsection{\tt grad(g,s,c,t)}

The arguments of this function are all pointers to double, except time \verb+t+, which is a %%@
double; \verb+s[]+ is the state variable array; \verb+c[]+ is the array of model constants; %%@
\verb+g[]+ is the array in which the user must return the time derivatives of the state %%@
variables. Elements of \verb+s[]+ must not be altered in this function. It is very unlikely to be %%@
a good idea to change elements of \verb+c[]+ in this function.

This is the routine in which the user specifies the gradients of the state variables at time $ %%@
t$, given the the state variables at time $ t$ (supplied in \verb+s[]+), the model constants, and %%@
(optionally) lagged values of the state variables (or lagged functions of state variables). For %%@
each state variable \verb+s[i]+ you must supply a value in \verb+g[i]+. For example if the %%@
equation for the gradient of state variable 0 is:
$$
\dif{s_0}{t} = c_0 s_0 - c_1 s_0 s_1
$$
then the corresponding line in the routine \verb+grad()+ would be:\\
\verb+g[0]=c[0]*s[0]-c[1]*s[0]*s[1];+\\

Lagged variables are accessed using the routines \verb+pastvalue()+ and (occasionally) %%@
\verb+pastgradient()+. These are described later, but for completeness an example of their use is %%@
given here. Suppose that you want to code the equation:
$$
\dif{s_0}{t} = \left \{ \begin{array}{lc} c_0s_0(0) & t<c_1 \\ c_0s_0(t-c_1) & t\ge c_1 %%@
\end{array} \right .
$$
subject to the initial condition $s_0(0)=c_2$. Assuming that the $0^{th}$ history variable (see %%@
\verb+storehistory()+ and \verb+pastvalue()+, below) contains the lagged values of \verb+s[0]+ %%@
then the appropriate code would be:
\begin{verbatim}
if (t<c[1]) g[0]=c[0]*c[2];
else g[0]=c[0]*pastvalue(0,t-c[1],0);
\end{verbatim}
\noindent{\bf Notes:}
\begin{enumerate}
\item If the gradient changes discontinuously, then it is often good practice to specify a switch %%@
at the time of the discontinuity: this forces the integration to step to exactly the point of %%@
discontinuity, before continuing, which ensures that the continuity assumptions of the integrator %%@
are not violated (the specified switch should leave all state variables unchanged).
\item The specification of your model should never involve writing to global variables from %%@
within \verb+grad()+: \verb+grad()+ is called multiple times per step of the integrator, and %%@
\verb+t+ may increase {\it or decrease} between calls. Similarly writing to static variables is %%@
very rarely appropriate as part of model specification. 
\item \verb+g[]+ contains meaningless values on entry to the function.  
\end{enumerate} 

\subsection{\tt storehistory(his,ghis,g,s,c,t)} 

The arguments of this function are all pointers to double except for time \verb+t+ which is a %%@
double. \verb+g[]+ and \verb+s[]+ contain the current values of the gradients of the state %%@
variables (w.r.t. time) and the state variables; their elements must not be altered by this %%@
function. \verb+c[]+ is the array of model constants. \verb+his[]+ and \verb+ghis[]+ are arrays %%@
in which you should store the value and time derivative, at time \verb+t+, of any quantity that %%@
you want to use as a lagged variable within your model.  

If your model equations depend on lagged quantities then these must be stored. %%@
\verb+storehistory()+ is the function that allows you to do this. The lagged variables (or %%@
``history variables'') are stored only at discrete times. Interpolation is necessary to estimate %%@
the values of lagged variables between these storage times. In order to ensure correct adaptive %%@
stepping for integration this interpolation must be performed to a higher order of accuracy than %%@
the integration. To achieve this requires that both the values {\it and time derivatives} of the %%@
state variables be stored\footnote{This consistency of integrator and interpolator is solv95's %%@
chief advantage over other DDE solving packages.}. The lagged values and gradients for each %%@
history variable are stored in a ringbuffer for access by \verb+pastvalue()+ or occasionally %%@
\verb+pastgradient()+ (see below). 

As an example, if your model is the simple DDE:
$$
\dif{s_0}{t} = \left \{ \begin{array}{lc} c_0s_0(0) & t<c_1 \\ c_0s_0(t-c_1) & t\ge c_1 %%@
\end{array} \right .
$$
then history variable 0 would be defined as $s_0$. The appropriate piece of code within %%@
\verb+storehistory()+ would be:
\begin{verbatim}
his[0]=s[0];
ghis[0]=g[0];
\end{verbatim}
and the within \verb+grad()+ the $0^{th}$ state variable would be used to specify the gradient of %%@
$s_0$ as follows:
\begin{verbatim}
if (t<c[1]) g[0]=c[0]*c[2];
else g[0]=c[0]*pastvalue(0,t-c[1],0);
\end{verbatim}

History variables do not have to be state variables themselves (although it's usually easiest to %%@
set models up that way), for example if history variable 1 was to be $ e^{s_0}+ 2 s_1$ then the %%@
appropriate code in \verb+storehistory()+ would be:
\begin{verbatim}
his[1]=exp(s[0])+2*s[1];
ghis[1]=g[0]*exp(s[0])+2*g[1];
\end{verbatim}

If you must write to global variables, this routine is a reasonable place to do it, but be aware %%@
that you may compromise the accuracy of numerical model solution (since the adaptive stepping %%@
algorithm has no way of `seeing' this variable).

\subsection{\tt initstate(s,c,t)}

The arguments of this routine are: (pointer to double) \verb+s+, the state variable array; %%@
(pointer to double) \verb+c+ the constant array; (double) \verb+t+ the time at start of model %%@
integration.

This function is where the initial values of all state variables must be set by the user. For %%@
example to set state variable 0 to $ c_3$ and state variable 1 to 3.45, you would use the %%@
following code:
\begin{verbatim}
s[0]=c[3];
s[1]=3.45;
\end{verbatim}
It is quite safe to modify global variables and elements of \verb+c[]+ within this function.

\subsection{\tt statescale(double *scale)}

Solv95 controls integration error by specifying that it should be less than some specified %%@
proportion of each state variable. This can cause difficulties when some state variables are in %%@
the vicinity of zero. For example, if a state variable is at zero, but about to leave that state, %%@
then the integrator may attempt to estimate that variable with zero error - which requires a zero %%@
timestep! The solution adopted is to get the user to specify a small number to be added to the %%@
magnitude of each state variable when estimating proportianal integration error. These numbers %%@
(one for each state variable) must be supplied by the user in array \verb+scale[]+. 

Any elements of \verb+scale[]+ that are unspecified are taken as zero.  

\subsection{\tt initout(usercontrol *out)}
 
This routine is where you provide various initialization information for setting up the model and %%@
the display of the model's solutions. The routine is used to fill out the structure \verb+out+: %%@
there is a large amount to specify, but it is straightforward to do so. The following sections %%@
cover the  elements of \verb+out+: any element marked by a * is optional.
\subsubsection{Model control constants and defaults}
\begin{itemize}
\item \verb+out->nhv+ the (integer) number of history variables in the model.
\item \verb+out->hbsize+ the (integer) number of lagged values to store for each history %%@
variable, the larger this number, the larger the model time lags can be.
\item \verb+out->nlag+ this is the (integer) number of distinct lags at which each particular %%@
history variable is to be accessed. For example if you want to access history variable 0 at lag %%@
$t-12.0$, and history variable 1 at $t-3.0$, $t-20.0$ and $t-13.2$, then you would set %%@
\verb+out->nlag=3+. All \verb+out->nlag+ does is to control the number of distinct place markers %%@
used in the history buffer for each state variable - the place markers are used to save time when %%@
searching for the right place in the history buffer. This feature is purely to allow efficient %%@
access of the history buffer - so you can always set \verb+out->nlag=1+, it's just that this may %%@
be a bit inefficient. 
\item \verb+out->nsw+ the number of switches (which is also the number of switch functions). 
\item \verb+out->dout+ the output timestep - determines the approximate time between outputting %%@
results to windows and any specified file.  
\item \verb+out->tol+ the integration tolerance - that is the maximum error tolerated at each %%@
timestep (as a proportion of the state variable concerned).
\item \verb+out->dt+ the default initial timestep this isn't too crucial, as the timestep will be %%@
rejected if it's too big. But note that the maximum timestep is set to $100 \times {\tt out->dt} %%@
$ and the minimum to $1 \times 10^{-9} \times {\tt out->dt}$ (I know, I know - you can change it %%@
in {\tt dde()} in file ddeq.c.)
\item \verb+out->t0+ the default integration start time.
\item \verb+out->t1+ the default integration end time.
\end{itemize} 
\subsubsection{constants {\tt c[]}}
\begin{itemize}
\item \verb+out->c[i]+ the default value for the $i^{th}$ constant \verb+c[i]+ $i $ (starts at %%@
zero).
\item[*] \verb+out->cname[i]+ name for the $i^{th}$ constant. e.g. to set the name of constant 0 %%@
to ``fred'' use the line \verb+out->cname[0]="fred";+.
\item[*] \verb+out->cinfo[i]+ extra information about a constant that can be displayed at run %%@
time by clicking on the constant's name. e.g. 
\begin{verbatim}
  out->cinfo[0]="This constant measures the rate of accumulation of \
frustration when writing documentation";
\end{verbatim}
\end{itemize} 
\subsubsection{State variable labels and other information}
\begin{itemize}
\item[*] \verb+out->label[i]+ label for the $i^{th}$ state variable. e.g.
       \verb+out->label[1]="population of adults"+. It's only worth supplying these for state %%@
variables that you intend to output.
\item[*] \verb+out->initialtitle+ the title to be displayed when solv95 starts.
\item[*] \verb+out->initialtext+ explanatory text to be displayed when solv95 starts.
\end{itemize}
\subsubsection{Windowed output}
Solv95 allows the user to control how many windows output will be displayed in, as well as how %%@
many and which state variables will be displayed in each window.
\begin{itemize}
\item[*] \verb+out->xlabel+ the label to be attached to the `time' axis of each output plot.
\item \verb+out->no_windows+ the (integer) number of windows for displaying output.
\item \verb+out->lines[i]+ the number of state variables to be plotted in the $i^{th}$ window.
\item \verb+out->index[i].win+ the index of the window in which the $i^{th}$ state variable is to %%@
be displayed - only supply this for those state variables that are to be displayed.
\item \verb+out->index[i].cur+ the index of the curve within a particular window which will %%@
display state variable i. e.g. to display state variable 0 as curve 1 of window 2 use the lines:
\begin{verbatim}
out->index[0].win=2;
out->index[0].cur=1;
\end{verbatim}
\item[*] \verb+out->wname[i]+ the name of the $i^{th}$ window (displayed in the window header %%@
bar).
\item \verb+out->range[i].y0+ the initial minimum y axis value for window i.
\item \verb+out->range[i].y1+ the initial maximum y axis value for window i.
\end{itemize}
\subsubsection{File output}
A text output file can be produced, it's columns are time (or whatever variable is being %%@
integrated with respect to) followed by those state variables specified for output. 
\begin{itemize}
\item[*] \verb+out->fileno+ the integer number of state variables to output.
\item[*] \verb+out->fout[i]+ the index of the state variable to be output in column $i+1$ of the %%@
output file ($i$ starts at zero, column zero of the output file is time). 
\end{itemize}
\subsection{\tt pastvalue(i,t,j)}
This function is called by the user to access history variables at lagged times. \verb+i+ is the %%@
index of the history variable (starting at zero); \verb+t+ is the (double) time at which the %%@
history variable is to be evaluated; \verb+j+ is the index of the history buffer location marker %%@
being used at this call. The function returns a double. Examples of the use of this function are %%@
given in the sections on \verb+storehistory()+ and \verb+grad()+ above, so here a slightly more %%@
complicated example is given.

Consider the contrived situation of a population that produces half its offspring in an %%@
environment promoting fast maturation  and half in an environment promoting slow maturation, %%@
suppose also that the maturation time varies throughout the year. Suitable equations might %%@
be:
\begin{eqnarray*}
\dif{s_0}{t} & = & \frac{1}{2} c_0 s_0 (t-\tau_0) + \frac{1}{2} c_0 s_0 (t-\tau_1) - c_1 s_0(y) %%@
\\ 
\tau_0 &=& 20+10 \sin(2 \pi (t-90)/365)\\
\tau_1 &=& 2\tau_0 
\end{eqnarray*}
Suitable code for this in \verb+grad()+ might be:
\begin{verbatim}
. 
. 
T0 = 20.0+10.0*sin(6.28318530718*(t-90.0)/365.0);
T1 = 2*T0;
if (t>T1)
g[0]=pastvalue(0,t-T1,0)*0.5*c[0]+pastvalue(0,t-T2,1)*0.5*c[0]-c[1]*s[0];
else
.
.
\end{verbatim}  
Note the important point that you can not request lagged variables from before the start of %%@
integration - instead you have to define your model in such a way that it does not require such %%@
values: this is always possible for a well defined model.

\subsection{\tt pastgradient(i,t,j)}

This is exactly the same as \verb+pastvalue()+ except that it returns the time derivative of the %%@
lagged variable - its use is somewhat dubious, since the order of approximation of the %%@
interpolated gradients is lower than the order of approximation of the integrator which may lead %%@
to infelicities in the numerical solution of models using this function.

\subsection{Coding hints}
\begin{itemize}
\item One  use of switches, which is not obvious, is to make sure that the integrator ``sees'' %%@
short events - for example, if a population is at equilibrium it is possible for the integrator %%@
to be striding along taking such big steps that it steps right over a short pulse of immigrants - %%@
to get around this, you either need to modify the maximum timestep (see above) or put a switch in %%@
place somewhere during the pulse (which will force the integrator to stop at that point, thereby %%@
``noticing'' the pulse).  
\item You may want to write information to the screen from within your model - the easiest way to %%@
do this is to use a Windows ``messagebox''. Here is an example of a snippet of code that does %%@
just that:
\begin{verbatim}
char str[100];
.
.
sprintf(str,"Total host pop = %g",totpop);
MessageBox(HWND_DESKTOP,str,"Info!",MB_ICONEXCLAMATION|MB_OK);
\end{verbatim}
\end{itemize}
\section{Examples}
Solv95 comes with 3 examples model files: {\tt ode\_eg.c} is an ordinary differential equation %%@
model; {\tt dde\_eg.c} is a delay differential equation model; {\tt sdde\_eg.c} is a delay %%@
differential equation model with switches. The purpose of this section is simply to specify these %%@
3 models mathematically so that, in conjunction with the 3 files, the reader can see how the %%@
different types of model are coded up in practice.
\subsection{An ODE model}
Consider the classic Lokta- Volterra predator- prey model:
\begin{eqnarray*}
\dif{P}{t} & = & \alpha N P - \delta P \\
\dif{N}{t} & = & \beta N - \gamma N P 
\end{eqnarray*} 
where Greek letters are used for model parameters, $ N $ is prey population and $ P $ is predator %%@
population (and typically $ \gamma > \alpha$). The file {\tt ode\_eg.c} codes up this model %%@
defining: ${\tt s[0]}\equiv N$, ${\tt s[1]}\equiv P$, ${\tt c[0]}\equiv \alpha$, ${\tt %%@
c[1]}\equiv  \delta$, ${\tt c[2]}\equiv  \beta$, ${\tt c[3]}\equiv  \gamma$, ${\tt c[4]}\equiv %%@
N(0)$ and ${\tt c[5]}\equiv P(0)$.

\subsection{A DDE model}
Consider Gurney and Nisbet's (1981) model of Nicholson's (1954) famous\footnote{If you're an %%@
ecologist!} blowflies:
$$
\dif{A}{t}= \left \{ \begin{array} {lc}
- \delta A(t) & t < \tau \\
P A(t-\tau) e^{-A(t-\tau)/A_0} - \delta A(t) & t \ge \tau \end{array} \right .
$$
Where $ A $ is the population of adult flies in a laboratory population, $ \tau$ is the %%@
development time from egg to adulthood, $ P$ is maximum adult fecundity rate multiplied by the %%@
survival rate from egg to adult, and $ A_0$ is a parameter determining how quickly fecundity %%@
declines with adult population. 

This model is coded in the file \verb+dde_eg.c+.

\subsection{A switched DDE model}

This model (again vaguely ecological, but very simplistic for clarity) describes a resource %%@
replenished at regular intervals, grazed down by a population of consumers. It is assumed that %%@
consumers produce offspring in proportion to how much they eat, but that there is a lag between %%@
offspring production and those offspring starting to feed. 
The resources $ R$ are governed by:
$$
\dif{R}{t} = - \alpha R(t)C(t)
$$
except every $ T$ time units, when a quantity  $R_A$ is added to $R$. The consumer equation is:
$$
\dif{C}{t}= \left \{ \begin{array}{lc} 
- \delta C(t) & t < \tau \\
\gamma C(t-\tau)R(t-\tau) - \delta C(t) & t \ge \tau \end{array} \right . 
$$ 
where $ \gamma $ is the consumer fecundity per unit of resource (over $ \alpha$, strictly), $ %%@
\tau $ is the development time and $ \delta $ the {\it per capita} death rate.

The file \verb+sdde_eg.c+ codes up this model, with the assumption $ C(0)=1$ and $ R(0)=0$.

\section{The algorithm}

The method used for integration is an embedded RK2(3) scheme due to Fehlberg, and reported on %%@
page 170 of Hairer {\it at al.} (1987). Lagged variables (and gradients) are stored in a %%@
ringbuffer at each step of the integrator. Interpolation is required to estimate values of the %%@
lagged variables between storage times.  For numerical probity it is essential that the %%@
interpolation of lagged variables is of a higher order of approximation than the integrator, %%@
otherwise the assumptions underlying the error estimate from the RK pair will not be met. The %%@
algorithm used in Solv95 uses cubic hermite interpolation (e.g. Burden and Faires 1987) to %%@
achieve this (which is the reason that gradients need to be stored along with lagged values). The %%@
consequences of not using consistent interpolation and integration schemes are vividly %%@
illustrated in Highman (1993). Paul (1992) was also influential in the design of the method used %%@
here, and the step size selection is straight out of Press {\it et al.} (1992) (method, not %%@
code!). The RK2(3) pair used is not actually optimal - it should be possible to derive an %%@
improved scheme - see Butcher (1987) for an explanation of how to go about it. 

A deficiency of the algorithm is that switches are not tracked automatically -  if a switch %%@
impacts on a lagged variable the resulting derivative discontinuities are only dealt with by %%@
adaptive timestepping through them (Note, however, that lagged variables and gradients are stored %%@
immediately before and immediately after the application of a switch - i.e. the interpolator is %%@
not applied blindly through switches!). 

\begin{itemize}
\item[] Burden, R.L. and J.D. Faires (1985) Numerical Analysis. Pridle Weber and Schmidt, Boston.
\item[] Butcher, J.C. (1987) The Numerical Analysis of Ordinary Differential   
 Equations. John Wiley \& sons, Chichester.     
\item[] Hairer, E., S.P.Norsett \& G.Wanner (1987) Solving Ordinary differential 
                   Equations I. Springer-Verlag Berlin. p170 RKF2(3)B
\item[] Highman, D.J. (1993) Appl. Numer. Math. 12:403-414
\item[] Paul, C.A.H (1992) Appl. Numer. Math. 9:403-414
\item[] Press {\it et al.} (1992) Numerical Recipes in C. CUP 
\end{itemize}
  
\end{document}
